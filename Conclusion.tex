\chapter{Conclusion}\label{conclusion}
This thesis presented a literature review to find out what affects the sustainability of the software, how current agile methods account for sustainability, and how to measure the sustainability of the software. A new model for developing more sustainable software was created based on the current model used at a software development company called Kvanttori by adding relevant methods and metrics from the literature review. This model was then validated with existing criteria for sustainable development processes found in literature and expert interviews from developers at Kvanttori as well as external experts on sustainable software.

The literature review identified multiple relevant methods for increasing the sustainability of the software including the choices of architectural patterns, technologies, and hosting platforms. The literature review also found existing agile models for sustainable software which included sustainability-enhancing steps such as disposal of software, requirements engineering, and measuring energy consumption. Finally literature review identified multiple metrics and measurement tools for measuring different aspects of sustainability including easy-to-set-up tools for server environments for measuring the energy usage of different software applications running on the server. These findings combined with the existing methods used at Kvanttori produced a model that accounts for sustainability in different parts of the software development project and based on the expert interviews, should allow for making more sustainable software.

The model presented in this thesis can be used by software development companies to better implement methods to increase the sustainability of software in their development processes. The model can also be further specialized depending on the domain of software development it is used in such as embedded systems, mobile or native applications. Some additions need to be made when moving the model from a project level to an organizational level such as taking the larger sustainability impact including the social and individual aspects into account.