\keywords{green code, green software, sustainable software, agile, scrum}
\begin{abstract}
There is growing interest in developing software that is more sustainable technically and economically but also environmentally. The sustainability of software has been researched for over a decade and many methods for creating more sustainable software have been discovered. However, most development processes employing these methods have stayed on a theoretical level and practical implementations are hard to find. There are also many open questions about measuring the sustainability of software. This thesis presents a practical implementation of a sustainable software development process that also allows for measuring relevant sustainability metrics. A literature review was used to identify methods for creating sustainable software, existing agile methodologies for sustainable software, and relevant metrics and tools for measuring the sustainability of software. These findings were then added to a development model that splits the software development process into pre-development, development, usage, and post-development phases to facilitate sustainability in different phases of a software development project and measure relevant metrics in these phases. This model was then validated with existing literature on criteria for sustainable software development processes and with expert interviews. The result of this thesis is a development model that should produce more sustainable software and allow for the measurement of different aspects of sustainability with relevant metrics.
\end{abstract}