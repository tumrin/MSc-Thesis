\appendix

\chapter{Original quotes in Finnish}\label{appendixa}

\section{Interviewee 1}
\subsection{}\label{i11}
"Meillä on toi, Tech leadina, toi Lead developer rooli ollu pitkää."
\subsection{}\label{i12}
"Että mitenkä näistä saa niinkun toisen tason mittareita. Että niitä niinkun. Tulee jakolaskua tai kertolaskua jonkun jutun kanssa, jolla voidaan sitten ruveta saamaan jotain vertailtavuutta niinkun eri softien yli."
\subsection{}\label{i13}
"Sillä vois olla sellanen niinku Green code expert, jota käytetään silloin kun on tarve, että vältetään se, että kaiken tiedon ei tarvi olla siinä tiimissä."
\subsection{}\label{i14}
"Kyllä mä haluaisin kokeilla. Siis se, että jokainen mallihan tehdään niinku kontekstiin. Niin toi ei ihan suoraan meille uppoa."
\subsection{}\label{i15}
"Että jossain kohtaa, kun uudempi rauta on modernempi rauta, energiatehokkaampaa,
milloin kannattaa ottaa se Embedded Emissions-isku ja vaihtaa se rauta sieltä alta."
\subsection{}\label{i16}
"Pintapuolisesti mittauksesta löytyi 400 paperia, tai muutaman vuoden vanha kirjallisuustyö, missä oli 400 paperia mittaamiseen liittyen ja yksikään niistä ei loppupelissä ollut sellainen universaali."
\subsection{}\label{i17}
"Tässä nyt on viisi oli niitä sustainability kulmia ja tässä on kolme plus toi velocity."

\section{Interviewee 2}
\subsection{}\label{i21}
"No ainutta on ehkä silleen, kun tässä on tosi paljon asioita, mitä tämä muistaa pitää tehdä tälleen, niin mä uskon, että tästä tulee astetta raskaampi prosessi vähintäänkin kuin joku normi-agile ehkä olisi.
Niin sitten se voi tarkoittaa sitä, että tälle on sitten oma käyttötarkoituksensa.
Esimerkiksi jos halutaan sellaista laadukasta softaa kehittää ja sellaista, mitä tullaan käyttämään tälleen.
Mutta sitten jos tehdään vaikka jotain nopeita MVPtä tai muita.
Niin silloin mä, no se ei nyt ole varmaan tämän tarkoituskaan, niin siihen semmoinen ei varmaan silloin sovellu myöskään."
\subsection{}\label{i22}
"Joo, se voi toimia tähän sen takia, koska tämä ehkä vaatii sitten semmoista ainakin, vielä kun tämä ei nyt ole niin tämmöistä niin sanotusti widespread, tai tämmöistä niin laajasti osattua ehkä, taitoa tämmöinen sustainable kehitys, niin sitten on hyvä, että on semmoinen joku, joka tietää ne asiat."
\subsection{}\label{i23}
"Niin, en mä tiedä, mun mielestä se on ainoa toi ehkä toi social-puoli, että saisiko siihen jotain vielä mietittyä."
\subsection{}\label{i24}
"Sitten se paperillahan, jos ihmiset seuraa pointista pointtiin näitä asioita, niin kyllähän sen pitäisi tuottaa silloin kestävämpää."

\section{Interviewee 3}
\subsection{}\label{i31}
"No siis toi on tietysti, että jos sieltä saadaan sitä niinkun energiankulutustietoa, niin se on semmosta, mikä ei siis missään varmaan käytetä, kun ei sitä oo saatu.
Siis se on niinkun semmonen yksittäinen ihan niinkun uusi juttu."
\subsection{}\label{i32}
"Siis jos me puhutaan backlogista, niin mitä nyt on backlogia nähnyt, niin jos se on vaan se määrä, et kuinka monta kappaletta siellä on, niin nehän voi olla yks voi olla helvetin iso ja yks voi olla helvetin pieni."
\subsection{}\label{i33}
"Mä oon aina haaveillu semmosesta, et samalla kun sä pystyt tuolla nykyisillä valvontavehkeillä jäljittää sen yhden käyttäjäkliksun, niinku sä et tehä sen full stack-tracing, et kun se menee sinne kantaan astaan se kysely, niin se näkyy, missä se siellä juoksee ja menee, niin ihan samalla pystys jäljittää sen yhden käyttäjätoiminteen käyttävän energiamäärän, et se voitais viedä sille tasolle."
\subsection{}\label{i34}
"Et toi Scrum Masteri on mulle vähän niinku...Et mä en oo ihan varma, et miks sitä niinku tarvitaan."

\section{Interviewee 4}
\subsection{}\label{i41}
"Joo, taisin olla sanomassa vaan sitä, että en ottaisi pois ja en ehkä näe mitään sellaista selkeää, mikä olisi mun mielestä ylimääräistä tässä."
\subsection{}\label{i42}
"Mutta se, mikä siinä oli hyvä, oli, että kaikki korkean tason arkkitehtuurivalinnat ja teknologiavalinnat ja muut ohjaavat kyllä siihen suuntaan tosi hyvin."
\subsection{}\label{i43}
"Eli tämmöisten energiahotspottien seuraamiseen ei ollut mitään työkaluja ehdotettua?"